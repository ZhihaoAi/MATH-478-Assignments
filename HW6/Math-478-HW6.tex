\documentclass[10pt]{report}

\usepackage{subcaption} % for subfigures
\usepackage{amsthm} % for QED
%\usepackage{algpseudocode} % for pseudo-code
\usepackage{mathtools} % for delimiter

\usepackage{listings} % for code
\lstset
{
	language=Matlab,
	frame=single,
	basicstyle=\footnotesize,
	captionpos=b,
	numbers=left,
	stepnumber=1,
	showstringspaces=false,
	tabsize=4,
	breaklines=true,
	breakatwhitespace=false,
}

\usepackage{float} % for figure [H]
\usepackage{booktabs} % for tabular
\usepackage{caption} % for \caption*
\usepackage[export]{adjustbox} % for valign=t
\usepackage{array} % for column type m
\usepackage{verbatim}
\usepackage{graphicx}
\graphicspath{ {imgs/} }
\usepackage{fancyhdr}
\usepackage{amssymb}
\usepackage{amsmath}

%%%%%% Pagination
\setlength{\topmargin}{-.3 in}
\setlength{\oddsidemargin}{0in}
\setlength{\evensidemargin}{0in}
\setlength{\textheight}{9.in}
\setlength{\textwidth}{6.5in}

%Title page
\newcommand{\hwTitle}{Homework \#5}
\newcommand{\hwCourse}{Numerical Differential Equations/Computational Mathematics II}
\newcommand{\hmwkClassInstructor}{Professor Shuwang Li}

\title{
	\vspace{2in}
	\textmd{\textbf{\hwCourse\\\hwTitle}}\\
	\vspace{0.3in}\large{\textit{\hmwkClassInstructor}}
	\vspace{3in}
}

%\title{Homework 1}
\author{\textbf{Zhihao Ai}}
\date{}

%Header setting. 
\pagestyle{fancy}
\fancyhead[L]{Zhihao Ai}
\fancyhead[C]{Math 478}
\fancyhead[R]{Homework 5}
%%%%%%

%Global setting.
%\everymath{\displaystyle}
\setlength\parindent{0pt}

%Custom general commands.
\newcommand{\ds}{\displaystyle}
\newcommand{\ts}{\textstyle}
\newcommand{\f}[1] {f\left(#1\right)}
\newcommand{\eva}[2] {\left. #1 \right|_{#2}}
\newcommand{\dintt}[4] {\int_{#1}^{#2} #3 d#4}

\newcolumntype{N}{ >$ c <$}
\newcolumntype{M}[1]{>{\centering\arraybackslash $}m{#1}<{$}}

\newcommand{\abs}[1] {\left| #1 \right|}

\DeclarePairedDelimiter\autoparen{(}{)}
\newcommand{\pa}[1]{\autoparen*{#1}}
\DeclarePairedDelimiter\autodvert{\Vert}{\Vert}
\DeclarePairedDelimiter{\floor}{\lfloor}{\rfloor}
\newcommand{\norm}[1]{\autodvert*{#1}}

\begin{document}

\maketitle

\section*{Question 1}
% Fundamental concepts on error control..
\begin{enumerate}
	\item 
	(Problem 6.1) Find the error constants for the Adams-Bashforth method (2.7) and for Adams-Moulton methods with $s = 2, 3$.
	
	\item 
	(Problem 6.2) Prove that the error constant of the s-step backward differentiation formula is $-\beta/(s + 1)$, where $\beta$ was defined in (2.14).
	
	\item 
	(Problem 6.4) Prove that the embedded RK pair
	\[
	\renewcommand{\arraystretch}{1.2}
	\begin{array}{c|ccc}
	0 &&&\\
	\frac{1}{2} & \frac{1}{2} &&\\
	1 & -1 & 2 &\\ \hline
	& 0 & 1 &\\ \hline
	& \frac{1}{6} & \frac{2}{3} & \frac{1}{6}
	\end{array}
	\]
	combines a second-order and a third-order method.
\end{enumerate}

\section*{Computer Assignment}
Consider a linear BVP: $y'' = y - x$ on $0<x<1$ and $y(0)=0, y(1)=0$. The exact solution to this problem is $y(x) = x + \frac{e}{e^2-1}(e^{-x}-e^x)$.
\begin{enumerate}
	\item 
	Implement the linear shooting method to solve the problem using 10 subintervals. compute the numerical errors for each node point.
	
	\item 
	Implement the finite difference method to solve the problem using 10 subintervals. compute the numerical errors for each node point.
	
	\item 
	Comment on what you get in (a) and (b).
	
	\item 
	Suppose we are solving a nonlinear BVP, what changes are you going to make to the routines you developed in (a) and (b)?
\end{enumerate}

\end{document}
