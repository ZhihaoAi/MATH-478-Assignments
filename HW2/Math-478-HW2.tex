\documentclass[10pt]{report}

\usepackage{subcaption} % for subfigures
\usepackage{amsthm} % for QED
%\usepackage{algpseudocode} % for pseudo-code
\usepackage{mathtools} % for delimiter

\usepackage{listings} % for code
\lstset
{
	language=Matlab,
	frame=single,
	basicstyle=\footnotesize,
	captionpos=b,
	numbers=left,
	stepnumber=1,
	showstringspaces=false,
	tabsize=4,
	breaklines=true,
	breakatwhitespace=false,
}

\usepackage{siunitx} % for scientific notation
% for `e' in scientific notation
\sisetup{output-exponent-marker=\ensuremath{\mathrm{e}}}

\usepackage{float} % for figure [H]
\usepackage{booktabs} % for tabular
\usepackage{caption} % for \caption*
\usepackage[export]{adjustbox} % for valign=t
\usepackage{array} % for column type m
\usepackage{verbatim}
\usepackage{graphicx}
\graphicspath{ {imgs/} }
\usepackage{fancyhdr}
\usepackage{amssymb}
\usepackage{amsmath}

%%%%%% Pagination
\setlength{\topmargin}{-.3 in}
\setlength{\oddsidemargin}{0in}
\setlength{\evensidemargin}{0in}
\setlength{\textheight}{9.in}
\setlength{\textwidth}{6.5in}

%Title page
\newcommand{\hwTitle}{Homework \#2}
\newcommand{\hwCourse}{Numerical Differential Equations/Computational Mathematics II}
\newcommand{\hmwkClassInstructor}{Professor Shuwang Li}

\title{
	\vspace{2in}
	\textmd{\textbf{\hwCourse\\\hwTitle}}\\
	\vspace{0.3in}\large{\textit{\hmwkClassInstructor}}
	\vspace{3in}
}

%\title{Homework 1}
\author{\textbf{Zhihao Ai}}
\date{}

%Header setting. 
\pagestyle{fancy}
\fancyhead[L]{Zhihao Ai}
\fancyhead[C]{Math 478}
\fancyhead[R]{Homework 2}
%%%%%%

%Global setting.
\everymath{\displaystyle}
\setlength\parindent{0pt}

%Custom general commands.
\newcommand{\ds}{\displaystyle}
\newcommand{\f}[1] {f\left(#1\right)}
\newcommand{\eva}[2] {\left. #1 \right|_{#2}}
\newcommand{\dintt}[4] {\int_{#1}^{#2} #3 d#4}

\newcolumntype{M}[1]{>{\centering\arraybackslash}m{#1}}
\newcolumntype{C}{M{3em}}
\newcolumntype{D}{M{5em}}

\DeclarePairedDelimiter{\paren}{(}{)}
\newcommand{\abs}[1] {\left| #1 \right|}

%Custom local commands
\newcommand{\varQa} {\sqrt{\frac{3}{5}}}

\begin{document}

\maketitle

\section*{Question 1.}
% Fundamental concepts on numerical integration based on interpolation.
% Problem 8, 13 of section 7.2.
\begin{enumerate}
	\item 
	(Section 7.2 Problem 8) Find the formula
	\[
	\dintt{0}{1}{f(x)}{x} \approx A_0 f(0) + A_1 f(1)
	\]
	that is exact for all functions of the form $f(x)=ae^x+b\cos{(\pi x/2)}$.
	
	Since $f(0) = a+b, f(1) = ae, \dintt{0}{1}{ae^x+b\cos{(\pi x/2)}}{x} = a(e-1)+\frac{2b}{\pi}$,
	\[
	a(e-1)+\frac{2b}{\pi} = A_0 (a+b) + A_1 ae
	\]
	Rearraging the equation, we get
	\[
	\begin{cases}
	A_0 + A_1 e = e - 1\\[1ex]
	A_0 = \frac{2}{\pi}
	\end{cases}
	\Rightarrow
	\begin{cases}
	A_0 = \frac{2}{\pi}\\[1ex]
	A_1 = 1 - \frac{1}{e} - \frac{2}{\pi e}
	\end{cases}
	\]
	
	\item 
	(Section 7.2 Problem 13) Determine values for $A$, $B$ and $C$ that make the formula
	\[
	\dintt{0}{2}{xf(x)}{x} \approx Af(0) + Bf(1) + Cf(2)
	\]
	exact for all polynomials of degree as high as possible. What is the maximum degree?
	
	Since there are 3 data points, the maximum degree is 2. It is exact for all polynomials of degree $\le$ 2, so by using method of undetermined coefficients we have
	\[
	\begin{cases}
	2 = \dintt{0}{2}{x\cdot 1}{x} = A\cdot 1 + B\cdot 1 + C\cdot 1\\[1em]
	\frac{8}{3} = \dintt{0}{2}{x\cdot x}{x} = A\cdot 0 + B\cdot 1 + C\cdot 2\\[1em]
	4 = \dintt{0}{2}{x\cdot x^2}{x} = A\cdot 0 + B\cdot 1 + C\cdot 4
	\end{cases}
	\Rightarrow
	\begin{cases}
	A = 0\\[1em]
	B = \frac{4}{3}\\[1em]
	C = \frac{2}{3}
	\end{cases}
	\]
\end{enumerate}

\section*{Question 2.}
% Fundamental concepts on Gaussian Quadrature.
% Problem 7(a) and 10(a), 22 of section 7.3
\begin{enumerate}
	\item 
	(Section 7.3 Problem 7(a)) Find a formula of the form
	\[
	\dintt{0}{1}{xf(x)}{x} \approx \sum_{i=0}^{n} A_i f(x_i)
	\]
	with $n=1$, that is exact for all polynomials of degree 3.
	
	Since $n=1$, let $q_2(x) = x^2 + bx + c$. For $q_2$ to be $w$-orthogonal to $\Pi_n$, consider $p_0 = 1$ and $p_1 = x$, $p_0, p_1 \in \Pi_n$ and then we have
	\[
	\begin{cases}
	\dintt{0}{1}{q_2(x)p_0(x)w(x)}{x} = \dintt{0}{1}{(x^2 + bx + c)\cdot 1\cdot x}{x} = 0\\[1em]
	\dintt{0}{1}{q_2(x)p_1(x)w(x)}{x} = \dintt{0}{1}{(x^2 + bx + c)\cdot x\cdot x}{x} = 0
	\end{cases}
	\Rightarrow
	\begin{cases}
	b = -\frac{6}{5}\\[1em]
	c = \frac{3}{10}
	\end{cases}
	\]
	Finding the zeros of $q_2$, solve
	\[
	x^2 - \frac{6}{5}x + \frac{3}{10} = 0
	\quad\Rightarrow
	\begin{cases}
	x_0 = \frac{6-\sqrt{6}}{10}\\[1em]
	x_1 = \frac{6+\sqrt{6}}{10}
	\end{cases}
	\]
	Since the formula is exact for $f\in \Pi_3$, use method of undetermined coefficients and we have
	\[
	\begin{cases}
	\frac{1}{2} = \dintt{0}{1}{x\cdot 1}{x} = A_0\cdot 1 + A_1\cdot 1\\[1em]
	\frac{1}{3} = \dintt{0}{1}{x\cdot x}{x} = A_0\cdot \frac{6-\sqrt{6}}{10} + A_1\cdot \frac{6+\sqrt{6}}{10}
	\end{cases}
	\Rightarrow
	\begin{cases}
	A_0 = \frac{9-\sqrt{6}}{36}\\[1em]
	A_1 = \frac{9+\sqrt{6}}{36}
	\end{cases}
	\]
	Therefore, the formula is
	\[
	\dintt{0}{1}{xf(x)}{x} \approx \frac{9-\sqrt{6}}{36}f\left(\frac{6-\sqrt{6}}{10}\right) + \frac{9+\sqrt{6}}{36}f\left(\frac{6+\sqrt{6}}{10}\right)
	\]
	
	\item 
	(Section 7.3 Problem 10(a)) If the integration formula
	\[
	\dintt{-1}{1}{f(x)}{x} \approx f(\alpha) + f(-\alpha)
	\]
	is to be exact for all quadratic polynomials, what value of $\alpha$ should be used? Answer the same question for all cubic polynomials.
	
	Since $n=1$, the $x_i$'s are roots of Legendre polynomial $p_2(x) = \frac{1}{2}(3x^2 + 1)$, which are $\pm \frac{\sqrt{3}}{3}$; therefore, $\alpha = \frac{\sqrt{3}}{3}$. The formula is exact for $f\in \Pi_3$.
	
	\item 
	(Section 7.3 Problem 22) Show how the Gaussian quadrature rule
	\[
	\dintt{-1}{1}{f(x)}{x} \approx \frac{5}{9}f\left(-\sqrt{\frac{3}{5}}\right) + \frac{8}{9}f(0) + \frac{5}{9}f\left(\sqrt{\frac{3}{5}}\right)
	\]
	can be used for $\textstyle \dintt{a}{b}{f(x)}{x}$. Apply this result to evaluate:
	\begin{enumerate}
		\item 
		$\dintt{0}{\pi/2}{x}{x}$
		
		Let $x = \frac{\pi}{4}(t+1), dx = \frac{\pi}{4}dt, t\in [-1, 1]$. Then 
		\begin{align*}
		\dintt{0}{\pi/2}{x}{x} 
		&= \dintt{-1}{1}{f\left(\frac{\pi}{4}(t+1)\right)\cdot\frac{\pi}{4}}{t} \\
		&\approx \frac{\pi}{4} \left[
		\frac{5}{9}f\left(\frac{\pi}{4}\left(-\sqrt{\frac{3}{5}}+1\right)\right) 
		+ \frac{8}{9}f\left(\frac{\pi}{4}\left(0+1\right)\right) 
		+ \frac{5}{9}f\left(\frac{\pi}{4}\left(\sqrt{\frac{3}{5}}+1\right)\right) 
		\right]\\
		&\approx 1.2337
		\end{align*}
		
		\item 
		$\dintt{0}{4}{\frac{\sin{t}}{t}}{t}$
		
		Let $t =2(x+1), dt = 2dx, x\in [-1, 1]$. Then 
		\begin{align*}
		\dintt{0}{4}{\frac{\sin{t}}{t}}{t}
		&= \dintt{-1}{1}{f(2(x+1))\cdot2}{t} \\
		&\approx 2 \left[
		\frac{5}{9}f\left(2\left(-\sqrt{\frac{3}{5}}+1\right)\right) 
		+ \frac{8}{9}f\left(2\left(0+1\right)\right) 
		+ \frac{5}{9}f\left(2\left(\sqrt{\frac{3}{5}}+1\right)\right) 
		\right]\\
		&\approx 1.7580
		\end{align*}
	\end{enumerate}
\end{enumerate}

\section*{Question 3.}
% Fundamental concepts on numerical differentiations.
% Problem 16 of section 7.1
(Section 7.1 Problem 16) Using Taylor series expansions, derive the error term for the formula
\[
f''(x) \approx \frac{1}{h^2}\left[f(x) - 2f(x+h) + f(x+2h)\right]
\]
When $h$ is small,
\begin{align}
	f(x+h) &= f(x) + hf'(x) + \frac{h^2}{2}f''(x) + \frac{h^3}{3!}f'''(\xi_1)\\
	f(x+2h) &= f(x) + 2hf'(x) + \frac{(2h)^2}{2}f''(x) + \frac{(2h)^3}{3!}f'''(\xi_2)
\end{align}
$(2) - (1)\times 2$, we get
\begin{align*}
	f(x + 2h) - 2f(x+h) &= -f(x) + h^2f''(x) + \frac{4}{3}h^3f'''(\xi_2) - \frac{1}{3}h^3f'''(\xi_1)\\
	f''(x) &= \frac{1}{h^2}[f(x) - 2f(x+h) + f(x + 2h)] + \frac{1}{3}hf'''(\xi_1) - \frac{4}{3}hf'''(\xi_2) 
\end{align*}
Therefore, the error term
\[
	\epsilon = \abs{f''(x) - \frac{1}{h^2}[f(x) - 2f(x+h) + f(x + 2h)]} = \abs{\frac{1}{3}hf'''(\xi_1) - \frac{4}{3}hf'''(\xi_2)}
\]

\section*{Computer  Assignment}
Use composite trapezoidal rule and composite Simpson's rule to evaluate $\dintt{0}{1}{\frac{\sin{x}}{x}}{x}$. You can use $x_k = kh, h = 1/8 \text{ for } k = 0,1,2,\dots,8.$
\lstinputlisting{composite_eval.m}
The script above outputs 0.9457 when using composite trapezoidal rule and 0.9461 when using composite Simpson's rule.

\end{document}


